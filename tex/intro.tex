\documentclass[../main/thesis.tex]{subfiles}
\begin{document}
\newpage
%% Introduction

\chapter{Introduction}
\label{intro}
\section{Background and motivation}
\label{i-background}
In 2013 the Norwegian government started a project to build centers for cancer treatment using proton radiotherapy. The main reason for this is that radiotherapy using protons does less damage to healthy tissue than the more conventional radiotherapy using photons that Norway has today. Many photon radiotherapy patients develops cancer again a few decades after the original treatment as a result of the treatment radiation, which makes photon radiotherapy little suited for treating cancer in children. Every year 1000 to 1500 Norwegians will experience fewer side effects by being treated with protons instead of photons. \citep{uio2012} When treating cancer with protons it is critical to be able to deposit the energy of the radiation at the correct position inside the patient, both to kill the tumor and to avoid unnecessary damage to healthy tissue around the tumor. More on this in section \ref{t-proton}.

%To be able to assure the quality of the treatment system, the hospitals will need a system that can predict how the energy from a beam of protons will be absorbed in the body. For this and other purposes, SINTEF is developing a silicon based radiation detector (see section \ref{t-detector}), named Si-3DMiMic, which mimics the response of biological tissue to ionizing radiation on a cellular and sub-cellular level. By measuring how the energy is distributed in the detector, the hospitals will have an idea about how a patient will react to the same radiation beam. \citep{sintef3dmimic}

To be able to assure the quality of the treatment system, the hospitals need to be able to predict how the energy from a beam of protons will be absorbed in the body. The uncertainties because of the lack of knowledge on this causes an increased risk of side effects for proton therapy patients. To bolster the research on the effects of radiation on humans, SINTEF is developing a silicon based radiation detector (see section \ref{t-detector}), named Si-3DMiMic, which mimics the response of biological tissue to ionizing radiation on a cellular and sub-cellular level. By measuring how the energy is distributed in the detector, researchers will will learn more about how a patient will react to the same radiation beam. \citep{sintef3dmimic}

\section{Goal of the thesis}
\label{i-goal}




\section{Structure of the thesis}
\label{i-structure}
\textbf{Chapter 1: Introducton}\\
Introduces the thesis, including background, motivation and goals.

\textbf{Chapter 2: Theory}\\
Important theory for full understanding of the thesis. Includes theory of radiation, radiation detection, radiotherapy




\section{Scientific environment}
\label{i-environment}
The author is supervised by Professor Kjetil Ullaland in the electronics and measurement science research group at the \gls{ift} at the \gls{uib}. The author is a master candidate for the same research group.

This project is a part of the subatomic physics research group at \gls{ift} and was assigned to the author by Professor Dieter Röhrich of this group. The author is working closely together with Röhrich's Ph.D. student Andreas Tefre Samnøy who shares some of the authors project objectives. Post doctor Kristian Smeland Ytre-Haugen at \gls{ift} leads an ongoing project that looks into microdosimetry and relative biological effectiveness of proton and heavy ion therapy. 

Haukeland?

SINTEF's department of Microsystems and Nanotechnology produces the pixel detector that this project is developed around. 

\end{document}