\documentclass[../main/thesis.tex]{subfiles}
\begin{document}
\newchapter

\chapter{Conclusions \& Outlook}
\label{conc}
%\addcontentsline{toc}{chapter}{Conclusions \& Outlook}

The work of this thesis on the 3DMiMic detectors suggests that SINTEF's fabrication process has been a success. The newest wafer shows many good I-V curves, and the newer wafers produced from the experiences of the measurements of this thesis should have some improvements. The detectors still need to be properly tested with radiation before any definite conclusions can be made. This work will be continued by Andreas Samnøy in his Ph.D. work, as well as by researchers at the University of Wollongong. 

The work on the prototype IDE1180 pre-amplifier \& shaper shows many deviations to the preliminary datasheet. Some differences were expected, as the preliminary datasheet was made from measurements directly on the \gls{asic} without a test \gls{PCB}. It might be possible to reduce many of the differences by designing a new PCB. The deviation in gain and dynamic range at the four gain settings between measurements and datasheet might suggest that the tests are executed on different revisions of the \gls{asic} design. The high dependence of the gain to input load capacitance shows that the IDE1180 might have problems with detectors with high capacitance and requirements for high gain. This should not be an issue for the 3DMiMic detectors that have relatively low capacitance. The noise measurements show that the input referred noise is five times higher than in the datasheet at low input capacitance, and 75~\% higher at high capacitance. This can partly be explained by extra noise from the \gls{PCB}, but it is difficult to do a direct comparison as it is unknown how IDEAS has calculated the input referred noise from the measured noise on the output. Crosstalk between channels is not an issue as it is comparative to the noise.

As the 3DMiMic detectors have not yet been tested in detail with radiation, the charge collection time is not know. Therefore the minimum requirement for shaping time is unknown. At shaping times of 1 and 2~$\mu$s it is possible to read out the biggest signals expected from the 3DMiMic detectors by using the IDE1180, while the dynamic range is not big enough at default settings and 500 ns shaping time. IDE1180 fails the requirements for 3DMiMic when it comes to the smallest signals. A \acrlong{MIP} is expected to deposit about 1000 electrons, while the lowest noise measured for IDE1180 is equivalent to about 3000 electrons on the input. IDE1180 usefulness for the 3DMiMic detectors will therefore depend on the importance of the smallest radiation particles to the measurements. If the IDE1180 is going to be used, it will be crucial to work on shielding it from noise. Depending on its radiation hardness, which has not yet been measured, it might be placed on the same PCB as the detector to reduce the input noise. It is also important to make sure that the output signal is not affect as seen in figure \ref{fig-box-pulse}. Note that the IDE1180 is designed for short shaping times and very low input load capacitance, and is therefore taken out of its element in many of these tests. 
%shaping time
%dynamic range, linearity
%noise faraday cage, load capacitance

%ortec+shaper
Because of the high noise levels of the other investigated circuits with a viable shaping time and dynamic range, I would recommend using the ORTEC 142A with the 3DMiMic for the time being. This requires the purchase of a fitting portable shaper. The 142A also fails the noise requirements of the 3DMiMic, but by a much smaller margin. The 142A datasheet specifies about 550 electrons of noise at 20 pF. This is lower than the 1000 electrons from a \gls{MIP}, but by such a small margin that it can be hard to reliably measure these particles. In the future it should be considered to either create a new dedicated readout system for 3DMiMic, or modify an existing system that is close to the 3DMiMic requirements. This could for example be IDE1180 or ALICE TPC PASA. 

%However, the only known electronic systems with a noise level of less than 100 electrons are made for bump-bonding and therefore cannot be used directly with the standard 3DMiMic layouts.

\end{document}